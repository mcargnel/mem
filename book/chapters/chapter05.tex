En esta sección se presentan tres casos de estudio para poner a prueba las técnicas descriptas: \textit{Airfoil Self-Noise}, \textit{Concrete Comprehensive Strength} y \textit{Wine Quality}. El análisis de cada conjunto de datos comienza con una breve exploración y limpieza, seguida de una búsqueda de hiperparámetros (\textit{grid search}) para minimizar el error de validación cruzada en el subconjunto de entrenamiento (80\% de los datos). Una vez optimizados, los modelos se evalúan en el conjunto de prueba restante para finalmente interpretar sus resultados mediante \textit{Permutation Feature Importance} (PFI) y \textit{Partial Dependence Plots} (PDP).

Se comparan cuatro tipos de modelos: regresión lineal, árboles de decisión, \textit{Random Forest} y \textit{Gradient Boosting Machines}. Su rendimiento se mide a través del Error Cuadrático Medio (RMSE), el Error Absoluto Medio (MAE) y el Coeficiente de Determinación ($R^2$).

La configuración de hiperparámetros explorada para los árboles de decisión incluye variaciones en la profundidad máxima (10, 20, 30) y los criterios de división (mínimo de observaciones para dividir: 2, 5, 10; mínimo por hoja: 1, 2, 4). Para los métodos de ensamble, \textit{Random Forest} y \textit{Gradient Boosting}, se ajusta el número de estimadores (de 100 a 2000), sumando en este último la tasa de aprendizaje o \textit{learning rate} (0.01, 0.1, 0.2).

Por último, se analiza la estabilidad de la importancia de variables (PFI) entre los cinco mejores modelos de cada caso. La consistencia en estos resultados sugiere robustez en la interpretación; de lo contrario, podríamos estar ante la presencia de \textit{Rashomon Sets}, donde modelos con capacidades predictivas similares ofrecen explicaciones contradictorias del fenómeno. El objetivo no es necesariamente obtener el modelo perfecto, sino ilustrar la aplicación práctica de las metodologías de interpretabilidad discutidas.

\section{\textit{Airfoil Self-Noise}}

El primer conjunto de datos proviene de la NASA y describe pruebas aerodinámicas y acústicas realizadas en un túnel de viento anecoico sobre diferentes secciones de palas \cite{airfoil_self-noise_291}. El objetivo es predecir la presión sonora escalada en decibelios (\textit{scaled-sound-pressure}) a partir de características físicas y operativas. La muestra consta de 1503 observaciones completas (sin valores faltantes) que incluyen perfiles NACA 0012 de distintos tamaños sometidos a diversas velocidades y ángulos de ataque. Las variables disponibles se detallan en la \autoref{tbl:airfoil_variables}.

\begin{table}[ht!]
    \centering
    \caption{Variables del conjunto de datos \textit{Airfoil Self-Noise} y su significado.}
    \label{tbl:airfoil_variables}
    \begin{tabular}{ll}
    \hline
    \textbf{Variable} & \textbf{Significado} \\
    \hline
    \textit{frequency} & Frecuencia en hertzios \\
    \textit{attack-angle} & Ángulo de ataque en grados \\
    \textit{chord-length} & Longitud de cuerda en metros \\
    \textit{free-stream-velocity} & Velocidad de flujo libre en metros por segundo \\
    \textit{suction-side-displacement-thickness} & Espesor de desplazamiento del lado de succión en metros \\
    \textit{scaled-sound-pressure} & Presión sonora escalada en decibelios (var dep) \\
    \hline
    \end{tabular}
\end{table}

El análisis de correlación inicial (\autoref{fig:airfoil-correlation}) reveló una fuerte asociación entre el espesor de desplazamiento (\textit{suction-side-displacement-thickness}) y el ángulo de ataque (\textit{attack-angle}). Dado que este último también presentaba una correlación considerable con la longitud de cuerda (\textit{chord-length}) pero menor influencia lineal sobre la variable objetivo, se optó por excluirlo del modelado para reducir la multicolinealidad.

\begin{figure}[ht!]
    \centering
    \includegraphics[width=1\textwidth]{/Users/mcargnel/Documents/mem/book/images/capitulo_5/airfoil_corr.pdf}
    \caption{Correlación de las variables del conjunto de datos \textit{Airfoil}.}
    \label{fig:airfoil-correlation}
\end{figure}

Tras el entrenamiento, los resultados en el conjunto de prueba (\autoref{tbl:air_predictions}) muestran que \textit{Gradient Boosting} supera significativamente a los demás modelos en todas las métricas evaluadas. En contraste, la regresión lineal presenta el peor ajuste, un patrón que, como se verá más adelante, se repite en los tres casos de estudio, lo cual es esperable dada la complejidad y posibles no linealidades de los fenómenos físicos estudiados.

\begin{table}[ht!]
    \centering
    \caption{Resultados del mejor modelo de cada tipo evaluados en el conjunto de testeo para \textit{Airfoil}.}
    \label{tbl:air_predictions}
    \begin{tabular}{lccc}
    \hline
    Modelo&RMSE&MAE&$R^2$ \\
    \hline
    Regresión Lineal&4.9766&3.9174&0.5056 \\
    Árbol de decisión&2.2708&1.7131&0.8971 \\
    \textit{Random Forest}&1.8258&1.3177&0.9335 \\
    \textit{Gradient Boosting}&1.4761&1.0337&0.9565 \\
    \hline
    \end{tabular}
\end{table}

Para comprender qué factores impulsan las predicciones del modelo de \textit{Gradient Boosting}, se recurre a la importancia de variables por permutación (PFI). La \autoref{fig:airfoil-permutation-importance} señala a la frecuencia y el espesor de desplazamiento como los predictores dominantes, seguidos por la longitud de cuerda y la velocidad de flujo libre. Es relevante notar que los cinco modelos con mejor desempeño coincidieron en este orden de importancia, otorgando robustez a esta conclusión (\autoref{tbl:imp_airfoil_self_noise}).

\begin{figure}[ht!]
    \centering
    \includegraphics[width=1\textwidth]{/Users/mcargnel/Documents/mem/book/images/capitulo_5/airfoil_perm_importance.pdf}
    \caption{Importancia de variables (PFI) para el modelo de \textit{Gradient Boosting} en el conjunto \textit{Airfoil}.}
    \label{fig:airfoil-permutation-importance}
\end{figure}

Profundizando en la relación entre las variables más importantes y la presión sonora, los gráficos de dependencia parcial (\autoref{fig:airfoil-partial-dependence}) ofrecen una visión detallada. En el caso de la frecuencia, se observa una relación inversa clara: los niveles de presión sonora son máximos (130--140 dB) en el rango de bajas frecuencias (0--2500 Hz) y descienden progresivamente hasta los 100--110 dB cerca de los 20000 Hz. El análisis centrado corrobora este efecto, mostrando variaciones de hasta $\pm 20$ dB respecto a la media. No obstante, la escasez de observaciones por encima de los 7500 Hz sugiere interpretar con cautela las predicciones en ese extremo del espectro.

Por su parte, el espesor de desplazamiento muestra un comportamiento no lineal más complejo. Aunque la presión sonora fluctúa principalmente entre 120 y 130 dB, la curva de respuesta presenta múltiples discontinuidades y una leve tendencia descendente general. Al igual que con la frecuencia, la distribución de los datos es asimétrica, concentrándose la gran mayoría de las observaciones por debajo de 0.015 metros.

\begin{figure}[ht!]
    \centering
    \includegraphics[width=1\textwidth]{/Users/mcargnel/Documents/mem/book/images/capitulo_5/airfoil_pdp.pdf}
    \caption{Gráficos de Dependencia Parcial (PDP) para \textit{frequency} y \textit{suction-side-displacement-thickness} en el modelo de \textit{Gradient Boosting} (\textit{Airfoil}).}
    \label{fig:airfoil-partial-dependence}
\end{figure} 
En síntesis, el análisis de este conjunto de datos demuestra la superioridad predictiva de los métodos de ensamble y permite identificar claramente a la frecuencia y el espesor de desplazamiento como los factores acústicos determinantes.

\section{\textit{Concrete Compressive Strength}}

El segundo caso de estudio utiliza el conjunto de datos \textit{Concrete Compressive Strength} \cite{concrete_compressive_strength_165}. Este consta de 1030 observaciones sobre la composición del concreto y su edad, con el fin de determinar su resistencia a la compresión. En la \autoref{tbl:concrete_variables} se describen las 9 variables involucradas:

\begin{table}[ht!]
    \centering
    \caption{Variables del conjunto de datos \textit{Concrete Compressive Strength} y su significado.}
    \label{tbl:concrete_variables}
    \begin{tabular}{ll}
    \hline
    \textbf{Variable} & \textbf{Significado} \\
    \hline
    Cemento & Cantidad de cemento en la mezcla (kg/m³) \\
    Escoria de alto horno & Cantidad de escoria de alto horno (kg/m³) \\
    Cenizas volantes & Cantidad de cenizas volantes (kg/m³) \\
    Agua & Cantidad de agua en la mezcla (kg/m³) \\
    Superplastificante & Cantidad de superplastificante (kg/m³) \\
    Agregado grueso & Cantidad de agregado grueso (kg/m³) \\
    Agregado fino & Cantidad de agregado fino (kg/m³) \\
    Edad & Edad del concreto en días \\
    Resistencia & Resistencia a la compresión del concreto (var dep) \\
    \hline
    \end{tabular}
\end{table}

A diferencia del caso anterior, el análisis de correlación (\autoref{fig:concrete_correlation}) no mostró coeficientes superiores a 0.8 en valor absoluto, por lo que se decidió conservar todas las variables predictoras para el modelado.

\begin{figure}[ht!]
    \centering
    \includegraphics[width=1\textwidth]{/Users/mcargnel/Documents/mem/book/images/capitulo_5/concrete_corr.pdf}
    \caption{Matriz de correlación entre las variables del conjunto de datos \textit{Concrete Compressive Strength}.}
    \label{fig:concrete_correlation}
\end{figure}

Al evaluar el rendimiento predictivo en el conjunto de prueba (\autoref{tbl:concrete_predictions}), \textit{Gradient Boosting} emerge nuevamente como el modelo más preciso, consolidando la tendencia observada previamente sobre la eficacia de los métodos de ensamble en estos contextos.

\begin{table}[ht!]
    \centering
    \caption{Resultados de los modelos en el conjunto de testeo para \textit{Concrete Compressive Strength}.}
    \label{tbl:concrete_predictions}
    \begin{tabular}{lccc}
    \hline
    Modelo&RMSE&MAE&$R^2$ \\
    \hline
    Regresión Lineal&9.7965&7.7456&0.6276 \\
    Árbol de decisión&7.0675&4.8364&0.8062 \\
    \textit{Random Forest}&5.5196&3.7957&0.8818 \\
    \textit{Gradient Boosting}&4.3500&2.8629&0.9266 \\
    \hline
    \end{tabular}
\end{table}

En cuanto a la interpretabilidad, el análisis de importancia de variables (\autoref{fig:concrete_permutation_importance}) identifica a la Edad y la Cantidad de Cemento como los factores más influyentes en la resistencia del concreto. Esta conclusión es consistente entre los cinco mejores modelos ajustados (\autoref{tbl:imp_concrete_strength}).

\begin{figure}[ht!]
    \centering
    \includegraphics[width=1\textwidth]{/Users/mcargnel/Documents/mem/book/images/capitulo_5/concrete_perm_importance.pdf}
    \caption{Importancia de las variables (PFI) del modelo \textit{Gradient Boosting} para \textit{Concrete Compressive Strength}.}
    \label{fig:concrete_permutation_importance}
\end{figure}

Los gráficos de dependencia parcial (\autoref{fig:concrete_partial_dependence}) permiten examinar en detalle el efecto de estas variables clave. Para la Edad, se aprecia un crecimiento pronunciado de la resistencia durante los primeros 50 días, tras lo cual la curva se estabiliza, indicando ganancias marginales más modestas con el paso del tiempo. Por su parte, la relación con el Cemento presenta un comportamiento escalonado con incrementos notables en umbrales específicos (250, 300 y 350 kg/m³). Un hallazgo interesante de los gráficos ICE es el paralelismo de las curvas individuales, lo que sugiere que el efecto de estas variables es aditivo y relativamente independiente de las interacciones con otros componentes de la mezcla.

\begin{figure}[ht!]
    \centering
    \includegraphics[width=1\textwidth]{/Users/mcargnel/Documents/mem/book/images/capitulo_5/concrete_pdp.pdf}
    \caption{Partial dependence plots (PDPs) e individual conditional expectation (ICE) para edad y cemento en el conjunto de datos \textit{Concrete Compressive Strength}.}
    \label{fig:concrete_partial_dependence}
\end{figure}

En resumen, este análisis confirma que la maduración del concreto (edad) y su contenido de cemento son los determinantes críticos de su resistencia, siendo \textit{Gradient Boosting} el algoritmo más capaz de capturar estas relaciones no lineales.

\section{\textit{Wine Quality}}

Finalmente, se analiza el conjunto de datos \textit{Wine Quality} \cite{wine_109}, que contiene 4898 muestras de vinos blancos de una región específica de Italia. Se dispone de 11 variables fisicoquímicas (\autoref{tbl:wine_variables}) para predecir la calidad sensorial del vino, medida en una escala discreta de 0 a 10.

\begin{table}[ht!]
    \centering
    \caption{Variables del conjunto de datos \textit{Wine Quality} y su significado.}
    \label{tbl:wine_variables}
    \begin{tabular}{ll}
    \hline
    \textbf{Variable} & \textbf{Significado} \\
    \hline
    \textit{fixed\_acidity} & Acidez fija en el vino (principalmente ácido tartárico) \\
    \textit{volatile\_acidity} & Acidez volátil en el vino (principalmente ácido acético) \\
    \textit{citric\_acid} & Ácido cítrico presente en el vino \\
    \textit{residual\_sugar} & Cantidad de azúcar residual en el vino \\
    \textit{chlorides} & Contenido de cloruros (sal) en el vino \\
    \textit{free\_sulfur\_dioxide} & Dióxido de azufre libre en el vino \\
    \textit{total\_sulfur\_dioxide} & Dióxido de azufre total en el vino (libre + combinado) \\
    \textit{density} & Densidad del vino \\
    \textit{pH} & Nivel de acidez/alcalinidad del vino (escala pH) \\
    \textit{sulphates} & Contenido de sulfatos en el vino \\
    \textit{alcohol} & Contenido de alcohol en el vino (\% por volumen) \\
    \textit{quality} & Calidad del vino, puntuación entre 0 y 10 (var. dep.)\\
    \hline
    \end{tabular}
\end{table}

Antes del modelado, se detectó una alta colinealidad entre el dióxido de azufre total y el libre (\autoref{fig:wine_corr}), procediéndose a eliminar este último para simplificar el modelo sin perder información significativa.

\begin{figure}[ht!]
    \centering
    \includegraphics[width=1\textwidth]{/Users/mcargnel/Documents/mem/book/images/capitulo_5/wine_corr.pdf}
    \caption{Matriz de correlación para las variables del conjunto de datos \textit{Wine Quality}.}
    \label{fig:wine_corr}
\end{figure}

Los resultados en el conjunto de prueba (\autoref{tbl:wine_predictions}) presentan un cambio notable respecto a los casos anteriores: aquí, \textit{Random Forest} obtiene la mejor, superando incluso a \textit{Gradient Boosting}. Aunque los valores de $R^2$ son en general más bajos que en los otros conjuntos de datos, lo cual es típico en problemas de calidad subjetiva con alta varianza, el bosque aleatorio logra explicar casi el 50\% de la variabilidad.

\begin{table}[ht!]
    \centering
    \caption{Resultados de los modelos en el conjunto de testeo para \textit{Wine}.}
    \label{tbl:wine_predictions}
    \begin{tabular}{lccc}
    \hline
    Modelo&RMSE&MAE&$R^2$ \\
    \hline
    Regresión Lineal&0.7378&0.5684&0.2630 \\
    Árbol de decisión&0.7697&0.5869&0.1979 \\
    \textit{Random Forest}&0.6103&0.4379&0.4957 \\
    \textit{Gradient Boosting}&0.6483&0.4818&0.4308 \\
    \hline
    \end{tabular}
\end{table}

La importancia de variables (\autoref{fig:wine_perm_importance}) destaca rotundamente al contenido de alcohol como el principal predictor de la calidad, seguido por la acidez volátil. Esta jerarquía se mantiene estable a través de los diversos modelos ajustados (\autoref{tbl:imp_wine_quality}).

\begin{figure}[ht!]
    \centering
    \includegraphics[width=1\textwidth]{/Users/mcargnel/Documents/mem/book/images/capitulo_5/wine_perm_importance.pdf}
    \caption{Importancia de variables (PFI) en el modelo de \textit{Random Forest} para \textit{Wine Quality}.}
    \label{fig:wine_perm_importance}
\end{figure}

El análisis de dependencia parcial (\autoref{fig:wine_partial_dependence}) arroja luz sobre la naturaleza de estas influencias. Para el alcohol, la relación con la calidad es positiva y no lineal, con un impacto mayor en el rango medio (11\%--13\%). Por el contrario, la acidez volátil penaliza la calidad de forma casi lineal: a medida que aumenta su concentración, la puntuación predicha disminuye consistentemente.

\begin{figure}[ht!]
    \centering
    \includegraphics[width=1\textwidth]{/Users/mcargnel/Documents/mem/book/images/capitulo_5/wine_pdp.pdf}
    \caption{Gráficos de Dependencia Parcial (PDP) e ICE para alcohol y acidez volátil en el modelo de \textit{Random Forest} (\textit{Wine Quality}).}
    \label{fig:wine_partial_dependence}
\end{figure}

En conclusión, la aplicación de estas técnicas en tres dominios dispares demuestra su versatilidad. Más allá de la precisión predictiva, donde los ensambles dominaron, herramientas como PFI y PDP permitieron traducir cajas negras matemáticas en conocimientos de dominio tangibles: la importancia de las frecuencias bajas en el ruido aerodinámico, el rol crítico del curado en el concreto, y la preferencia por vinos con mayor grado alcohólico y menor acidez volátil.
